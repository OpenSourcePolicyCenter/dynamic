\documentclass[letterpaper,12pt]{article}

\usepackage{threeparttable}
\usepackage{geometry}
\geometry{letterpaper,tmargin=1in,bmargin=1in,lmargin=1.25in,rmargin=1.25in}
\usepackage[format=hang,font=normalsize,labelfont=bf]{caption}
\usepackage{amsmath}
\usepackage{multirow}
\usepackage{array}
\usepackage{delarray}
\usepackage{amssymb}
\usepackage{amsthm}
\usepackage{lscape}
\usepackage{natbib}
\usepackage{setspace}
\usepackage{float,color}
\usepackage[pdftex]{graphicx}
\usepackage{pdfsync}
\usepackage{verbatim}
\usepackage{placeins}
\usepackage{geometry}
\usepackage{pdflscape}
\synctex=1
\usepackage{hyperref}
\hypersetup{colorlinks,linkcolor=red,urlcolor=blue,citecolor=red}
\usepackage{bm}


\theoremstyle{definition}
\newtheorem{theorem}{Theorem}
\newtheorem{acknowledgement}[theorem]{Acknowledgement}
\newtheorem{algorithm}[theorem]{Algorithm}
\newtheorem{axiom}[theorem]{Axiom}
\newtheorem{case}[theorem]{Case}
\newtheorem{claim}[theorem]{Claim}
\newtheorem{conclusion}[theorem]{Conclusion}
\newtheorem{condition}[theorem]{Condition}
\newtheorem{conjecture}[theorem]{Conjecture}
\newtheorem{corollary}[theorem]{Corollary}
\newtheorem{criterion}[theorem]{Criterion}
\newtheorem{definition}{Definition} % Number definitions on their own
\newtheorem{derivation}{Derivation} % Number derivations on their own
\newtheorem{example}[theorem]{Example}
\newtheorem{exercise}[theorem]{Exercise}
\newtheorem{lemma}[theorem]{Lemma}
\newtheorem{notation}[theorem]{Notation}
\newtheorem{problem}[theorem]{Problem}
\newtheorem{proposition}{Proposition} % Number propositions on their own
\newtheorem{remark}[theorem]{Remark}
\newtheorem{solution}[theorem]{Solution}
\newtheorem{summary}[theorem]{Summary}
\bibliographystyle{aer}
\newcommand\ve{\varepsilon}
\renewcommand\theenumi{\roman{enumi}}
\newcommand\norm[1]{\left\lVert#1\right\rVert}

\begin{document}

This document outlines how we can us the first order conditions of the household to estimate the utility weights on the disutility of work and the warm glow bequest.  

\section{Estimating the utility weight on the disutility of work, $\chi^{n}_{s}$}.

The household first order condition for the choice of hours worked yields:

        \begin{equation}\label{EqEulerLabGen}
      \begin{split}
        &(c_{j,s,t})^{-\sigma}\Biggl(w_t e_{j,s} - \frac{\partial T_{j,s,t}}{\partial n_{j,s,t}}\Biggr) = e^{g_y t(1-\sigma)}\chi^n_{s}\biggl(\frac{b}{\tilde{l}}\biggr)\biggl(\frac{n_{j,s,t}}{\tilde{l}}\biggr)^{v-1}\Biggl[1 - \biggl(\frac{n_{j,s,t}}{\tilde{l}}\biggr)\Biggr]^{\frac{1-v}{v}} \\
        &\qquad\qquad\qquad\qquad\qquad\qquad\qquad\qquad\qquad\forall j,t, \quad\text{and}\quad E+1\leq s\leq E+S \\
        &\qquad\text{where}\quad c_{j,s,t} = \left(1 + r_t\right) b_{j,s,t} + w_t e_{j,s}n_{j,s,t} + \frac{BQ_{j,t}}{\lambda_j\tilde{N}_t} - b_{j,s+1,t+1} - T_{j,s,t} \\
        &\qquad\text{and}\quad \frac{\partial T_{j,s,t}}{\partial n_{j,s,t}} = w_t e_{j,s}\biggl[\tau^I\bigl(F\hat{a}_{j,s,t}\bigr) + \frac{F\hat{a}_{j,s,t}CD\bigl[2A(F\hat{a}_{j,s,t})+B\bigr]}{\bigl[A(F\hat{a}_{j,s,t})^2+B(F \hat{a}_{j,s,t})+C\bigr]^2} + \tau^P\Biggr] 
      \end{split}
    \end{equation}
    
    To simplify notation a bit, let $w_{t}e_{j,s}=\tilde{w}_{j,s,t}$, which is defined as the hourly earnings of household of ability type $j$, age $s$, at time $t$.  Further, we can write derivative of the tax function, $ \frac{\partial T_{j,s,t}}{\partial n_{j,s,t}}$ as $\tau^{l}(y_{j,s,t})\tilde{w}_{j,s,t}$, where $\tau^{l}(y_{j,s,t})$ is the marginal tax rate on labor income for an individual with taxable income $y_{j,s,t}$.  Now we can write the FOC as:
    
            \begin{equation}\label{EqEulerLabGen}
      \begin{split}
        &(c_{j,s,t})^{-\sigma}\Biggl(\tilde{w}_{j,s,t}(1-\tau^{l}(y_{j,s,t})\Biggr) = e^{g_y t(1-\sigma)}\chi^n_{s}\biggl(\frac{b}{\tilde{l}}\biggr)\biggl(\frac{n_{j,s,t}}{\tilde{l}}\biggr)^{v-1}\Biggl[1 - \biggl(\frac{n_{j,s,t}}{\tilde{l}}\biggr)\Biggr]^{\frac{1-v}{v}} 
        \end{split}
        \end{equation}
        
        This problem is deterministic, but we can assume that their is some noise in the data, thus the data analog to the model FOC is: 

  \begin{equation}\label{EqEulerLabGen}
      \begin{split}
        &(c_{j,s,t})^{-\sigma}\Biggl(\tilde{w}_{j,s,t}(1-\tau^{l}(y_{j,s,t})\Biggr) - e^{g_y t(1-\sigma)}\chi^n_{s}\biggl(\frac{b}{\tilde{l}}\biggr)\biggl(\frac{n_{j,s,t}}{\tilde{l}}\biggr)^{v-1}\Biggl[1 - \biggl(\frac{n_{j,s,t}}{\tilde{l}}\biggr)\Biggr]^{\frac{1-v}{v}} = \varepsilon_{j,s,t}  
        \end{split}
        \end{equation}

And the moment condition for a GMM estimator for each $\chi^{n}_{s}$ would be:

  \begin{equation}\label{EqEulerLabGen}
      \begin{split}
        &\sum_{J}\sum_{T}\left[(c_{j,s,t})^{-\sigma}\Biggl(\tilde{w}_{j,s,t}(1-\tau^{l}(y_{j,s,t})\Biggr) - e^{g_y t(1-\sigma)}\chi^n_{s}\biggl(\frac{b}{\tilde{l}}\biggr)\biggl(\frac{n_{j,s,t}}{\tilde{l}}\biggr)^{v-1}\Biggl[1 - \biggl(\frac{n_{j,s,t}}{\tilde{l}}\biggr)\Biggr]^{\frac{1-v}{v}}\right] =\\
        &  \quad  \quad  \quad  \quad  \quad  \quad \sum_{J}\sum_{T}\varepsilon_{j,s,t}  = 0
        \end{split}
        \end{equation}
        
        Note that the above equation for each $s$ - and we can have more moment conditions if we wish to have an over-identified model.  We may also think about estimating the parameters of the ellipse via this method.
        
        To estimate, we need data on consumption, $c$, and labor supply, $n$, by lifetime income group, age, and year.  We can get this from the PSID - see \\
        \href{http://www.federalreserve.gov/pubs/feds/2007/200716/200716pap.pdf}{http://www.federalreserve.gov/pubs/feds/2007/200716/200716pap.pdf} for a document outlining the measurement of consumption from the PSID.  We can find $\tau^{l}(y_{j,s,t})$ by running the PSID observation through a tax calculator (e.g. the OSPC calculator).  The remaining parameters are calibrated elsewhere.
        
        
        
\section{Estimating the utility weight on the warm glow bequest motive, $\chi^{b}_{j}$}.

The household first order condition for the choice of savings yields:

    \begin{equation}\label{EqEulerSavGen}
      \begin{split}
        &(c_{j,s,t})^{-\sigma} = \rho_s\chi^b_j\bigl(b_{j,s+1,t+1}\bigr)^{-\sigma} + \beta(1-\rho_s)(c_{j,s+1,t+1})^{-\sigma}\Biggl[(1 + r_{t+1}) - \frac{\partial T_{j,s+1,t+1}}{\partial b_{j,s+1,t+1}}\Biggr] \\
        &\qquad\qquad\qquad\qquad\qquad\qquad\qquad\qquad\forall j,t,\quad\text{and}\quad E+1\leq s \leq E+S-1 \\
        &\qquad\text{where}\quad \frac{\partial T_{j,s+1,t+1}}{\partial b_{j,s+1,t+1}} = ...\\
        &\qquad\qquad r_{t+1}\Biggl(\tau^I(F\hat{a}_{j,s+1,t+1}) + \frac{F\hat{a}_{j,s+1,t+1}CD\left[2A(F\hat{a}_{j,s+1,t+1}) + B\right]}{\left[A(F\hat{a}_{j,s+1,t+1})^2 + B(F\hat{a}_{j,s+1,t+1}) + C\right]^2}\Biggr) ... \\
        &\qquad\qquad \tau^W(\hat{b}_{j,s+1,t+1}) + \frac{\hat{b}_{j,s+1,t+1}PHM}{\left(H\hat{b}_{j,s+1,t+1} + M\right)^2}
      \end{split}
    \end{equation}
    
    We can write derivative of the tax function, $ \frac{\partial T_{j,s,t}}{\partial b_{j,s,t}}$ as $\tau^{b}(y_{j,s,t})r_{t+1}$, where $\tau^{b}(y_{j,s,t})$ is the marginal tax rate on capital income for an individual with taxable income $y_{j,s,t}$.  Now we can write the FOC as:
    
            \begin{equation}\label{EqEulerLabGen}
        (c_{j,s,t})^{-\sigma} = \rho_s\chi^b_j\bigl(b_{j,s+1,t+1}\bigr)^{-\sigma} + \beta(1-\rho_s)(c_{j,s+1,t+1})^{-\sigma}\Biggl[(1 + (1-\tau^{b}_{j,s+1,t+1})r_{t+1}) \Biggr]
        \end{equation}
        
        This problem is deterministic, but we can assume that their is some noise in the data, thus the data analog to the model FOC is: 

  \begin{equation}\label{EqEulerLabGen}
      \begin{split}
         (c_{j,s,t})^{-\sigma} - \rho_s\chi^b_j\bigl(b_{j,s+1,t+1}\bigr)^{-\sigma} + \beta(1-\rho_s)(c_{j,s+1,t+1})^{-\sigma}\Biggl[(1 + (1-\tau^{b}_{j,s+1,t+1})r_{t+1}) \Biggr] = \varepsilon_{j,s,t}  
        \end{split}
        \end{equation}

And the moment condition for a GMM estimator for each $\chi^{b}_{j}$ would be:

  \begin{equation}\label{EqEulerLabGen}
      \begin{split}
        &\sum_{S}\sum_{T}\left[( (c_{j,s,t})^{-\sigma} - \rho_s\chi^b_j\bigl(b_{j,s+1,t+1}\bigr)^{-\sigma} + \beta(1-\rho_s)(c_{j,s+1,t+1})^{-\sigma}\Biggl[(1 + (1-\tau^{b}_{j,s+1,t+1}r_{t+1}) \Biggr]\right] = \\
        &    \quad\quad\quad\quad \quad \quad \sum_{S}\sum_{T}\varepsilon_{j,s,t}  = 0
        \end{split}
        \end{equation}
        
        Note that the above equation for each $j$ - and we can have more moment conditions if we wish to have an over-identified model. 
        
        To estimate, we need data on consumption, $c$, and wealth, $b$, by lifetime income group, age, and year.  We can get consumption from the PSID, as noted above.  We can also get wealth from the PSID, at least for the years 1984-2005 - see \href{http://www.brookings.edu/~/media/research/files/papers/2009/2/saving-wealth-bosworth/02\_saving\_wealth\_bosworth.pdf}{http://www.brookings.edu/~/media/research/files/papers/2009/2/saving-wealth-bosworth/02\_saving\_wealth\_bosworth.pdf} for a document outlining the measurement of wealth from the PSID.  We can find $\tau^{b}(y_{j,s,t})$ by running the PSID observation through a tax calculator (e.g. the OSPC calculator).  The remaining parameters are calibrated elsewhere.
        


   
    


\end{document}