	\documentclass[article,11pt,letterpaper,fleqn]{article}
\usepackage{graphicx,color}
\usepackage{array}
\usepackage{threeparttable}
\usepackage[format=hang,font=normalsize,labelfont=bf]{caption}
\usepackage{colortbl}
\usepackage{multirow}
\usepackage{geometry}
\usepackage{subfigure}
\geometry{letterpaper,tmargin=1in,bmargin=1in,lmargin=1.25in,rmargin=1.25in}
\usepackage{hyperref}
\hypersetup{colorlinks,%
citecolor=red,%
filecolor=red,%
linkcolor=red,%
urlcolor=blue,%
pdftex}
\usepackage{amsmath}
\usepackage{amssymb}
\usepackage{amsthm}
\usepackage{harvard}
\usepackage{tikz}
\usepackage{setspace}
\usepackage{float,graphicx,color}
\usepackage{appendix}
\usepackage{longtable}
\newtheorem*{thm}{Theorem}
\theoremstyle{definition}
\usepackage{lscape}
\numberwithin{equation}{section}
\newcommand{\cn}{\citeasnoun} % shortens command to cite as noun
\newcommand\ve{\varepsilon}


\title{Guide to Calibration of Firm Production Function Parameters in the OLG Dynamic Scoring Model}
\date{\today}



% make tables with smaller sized font 
\makeatletter
\def\table{\@ifnextchar[{\table@i}{\table@i[\fps@table]}}
\def\table@i[#1]{\@float{table}[#1]\footnotesize}
\makeatother



%\setlength{\topmargin}{-0.4in}
%\setlength{\topskip}{0.3in}    % between header and text
%\setlength{\textheight}{9.0in} % height of main text
%\setlength{\textwidth}{6in}    % width of text
%\setlength{\oddsidemargin}{39pt} %even side margin
%\setlength{\evensidemargin}{39pt} %odd side margin

\begin{document}
\bibliographystyle{aer}
\maketitle



\begin{abstract}
This will be the section in the dynamic scoring model handbook on calibrating production function parameters.
\end{abstract}

\section{Calibrating the Firm Production Functions}

Firm's combine capital, $K$, and effective labor, $EL$, with a fixed factor of production, $A$ to produce output, $X$.  We can think of the fixed factor of production as ``location specific capital".  It is fixed in the sense that its supply is perfectly inelastic.  It is location specific in the sense that it is proportional to the size of the population in the firm's home country at time $t$.  We write the amount of output produced as a function of this fixed factor and the value added, $VA$, from the input of capital and labor:

\begin{equation}
X_{t,m,c} = A_{t,m,c}(VA_{t,m,c})^{\alpha_{v,m}},
\end{equation} 

\noindent\noindent where the subscripts $t$, $m$, and $c$ refer to the model period, production industry, and production sector (corporate or non-corporate), respectively.  The parameter $\alpha_{v,m,c}$ is the share of output attributable to the firm's value added. The fixed factor of production is given by:

\begin{equation}
A_{t,m,c} = (A_{0,t,m,c}\omega_{m,c}N_{t})^{1-\alpha_{v,m}}
\end{equation}

\noindent\noindent  Thus the input from fixed factor of production used by the firm is given by the level of total factor productivity (TFP), $A_{0,t,m,c}$, and a exogenous share of the population, $N_{t}$ where the share is given by the parameter $\omega_{t,m,c}$ (\textcolor{red}{Not sure if we want this to vary by time, or just across production industry.}).  The share parameters must sum to one.  That is, $\sum_{m=1}^{M} \omega_{t,m,corp} + \sum_{m=1}^{M} \omega_{t,m,non-corp}= 1$.     We assume that TFP grows at the same rate across industry, with the growth rate given by $g_{a}$.  The value added is given by a CES function:

\begin{equation}
\label{eqn:prod_fun}
F(A_{0,t,m,c},K_{t,m,c},EL_{t,m,c})=VA_{t,m,c} =A_{0,t,m,c} \left[(\gamma_{m})^{1/\epsilon_{}}(K_{t,m,c})^{(\epsilon-1)/\epsilon_{}}+(1-\gamma_{m})^{1/\epsilon_{}}(e^{g_{y}t}EL_{t,m,c})^{(\epsilon_{m,c}-1)/\epsilon_{}}\right]^{(\epsilon_{}/(\epsilon_{}-1))},
\end{equation}

\noindent\noindent where $\epsilon$ gives the elasticity of substitution between capital and labor and $\gamma$ is the share parameter in the CES production function.  Effective labor units are affected by labor augmenting technological change.  The growth rate of this technology is give by $g_{y}$.  If $\alpha_{v}<1$, then the production function exhibits decreasing returns to scale with respect the firm's inputs of capital and labor.

This calibration will find values for the parameters $\epsilon$ (which we will assume to be the same across industry), $\gamma_{m}$ (which varies across industry), $\alpha_{v,m}$ (which varies across industry), and $\omega_{m,c}$ (which varies across industry and sector), and the growth rates $g_{y}$ and $g_{a}$ (which we assume to be the same across industry and sector).  We will also want to calibrate the value for total factor productivity across industry and sector in the initial model period, $A_{0,1,m,c}$.

To calibrate these parameters, we need data on output, employment (hours), capital, indirect business taxes, and labor taxes by industry and sector.

 Note that our treatment of sector will correspond to the tax-treatment of the business entity.  Therefore, we consider subchapter S corporations as non-corporate since they do not remit an entity level tax.  See Table \ref{tab:org_form} for this breakdown.  Note that these definitions are in contrast to the methodology used by BEA, where both subchapter C and subchapter S corporations fall into the ``corporate" sector and partnership and proprietorships fall under the non-corporate grouping.  

% Table generated by Excel2LaTeX from sheet 'SectorDefinitions'
\begin{table}[htbp]
  \centering
  \caption{Legal Form of Organization vs. Tax Treatment}
    \begin{tabular}{lll}
    \hline
    \hline
    Entity & Legal Form of Organization & Tax Treatment \\
   \hline
    C Corporation & Corporate & Corporate \\
    S Corporation & Corporate & Non-corporate \\
    Partnership & Non-corporate & n.a. \\
    \ \ \ Share of partnership income & n.a   & Corporate \\
    \ \ \ attributable to corporate partners & &  \\
    \ \ \ Share of partnership income& n.a.  & Non-corporate \\
    \ \ \ attributable to individual partners &  &  \\
    Sole Proprietorship & Non-corporate & Non-corporate \\
    \hline
    \hline
    \end{tabular}%
  \label{tab:org_form}%
\end{table}%


\section{Measuring Production Output by Industry}
\label{sec:step3}

To measure output as income by industry we use NIPA Tables 6.1B,C,D (not sure of the difference between these, so please look into it). From these income data, we want to subtract net interest income (Table 6.15 A, B, C, D (again, not sure of difference) so that we get just business income (and not financial income).  The measure of business income by industry is the measure of output.


\section{Measuring Employment by Industry}
\label{sec:step3}

To measure hours of employment by industry we use NIPA Tables 6.1B,C,D (not sure of the difference between these, so please look into it).   Note that these tables only give hours for employees.  We will want to impute hours for the self-employed as well.  Tables 6.7 A-D give the number of self-employed by industry.  \textcolor{red}{Think about this, but we should be able to get income to the self-employed by industry and make an assumption that they have the same wage rate and then be able to back out total self-employed hours by industry.}

Use Table 6.6 to get wages and salaries by industry.


\section{Measuring Capital Stock by Industry}
\label{sec:step3}

Done with the calibration of the depreciation parameters.

\section{Measuring Output and Indirect Business Taxes by Industry}
\label{sec:step3}

See \cn{FR1993}, Chapter 3 on the data for these taxes.  They use some unpublished worksheets from the Commerce Department's National Income Division (NID).  Perhaps we can find something similar.  Output taxes are:

\begin{enumerate}
\item Excise taxes
\item Indirect business taxes (defined below)
\item Less property taxes
\item Less motor vehicle taxes
\end{enumerate}

Indirect business taxes are:
\begin{enumerate}
\item Public utility fees
\item Severance fees
\item Occupancy fees
\item License fees
\item Other indirect business taxes
\item Non-tax payments to the government
\end{enumerate}

\section{Measuring Labor Taxes by Industry}
\label{sec:step3}

We'll want a gross of tax measure of labor income.  This is wags and salaries + employer contributions to social insurance + other labor income + the return to self-employed labor + half the contributions of the self-employed to social insurance .  Hopefully the first and fourth components are determined with the data on wages, as described above.  To get employer contributions to social insurance...  To get the contributions of the self-employed...




\section{Allocating Quantities Across Sectors}

Assume that wages within an industry are the same for the corporate and non-corporate firms.

To allocate quantities across corporate and non-corporate firms within an industry, we will assume that capital labor ratios are the same within an industry.  Thus, we will allocate the labor across firms in an industry in the same ratio as the capital stock is allocated across the corporate and non-corporate firms within that industry.  

Since the production function is homogenous of degree one (i.e., it displays constant returns to scale with respect to the capital, labor, and fixed factor inputs), the fact that the capital labor ratio is the same across firms in an industry means that that capital output ratio will be the same as well.  Thus, the ratio of capital in the corporate to non-corporate sector for each industry will be used to allocate output within the industry.

\subsection{A Note on Industry Classifications}

For our computational model, we would like to model the industries outlined in Table \ref{tab:prod_ind}.\footnote{This excludes the multi-national sector, which we still need to think about.}  These are mostly at the 2-digit NAICS classification level, with some exceptions for industries that may face special tax treatment.  The data sources do not all share the same level of industry detail.  For example, the BEA Detailed Fixed Asset Tables report fixed assets by asset type and by industry, where industry categories are generally at the 3-digit NAICS level.  IRS data is generally reported at the 2-digit NAICS level, with some items being available at finer levels of aggregation and others at more coarse levels.  BEA's Standard Fixed Asset Tables report fixed asset by industry, but only at a very coarse level.  

% Table generated by Excel2LaTeX from sheet 'ProductionIndustries'
\begin{table}[htbp]
  \centering
  \caption{Production Industries}
    \begin{tabular}{lll}
    \hline
    \hline
    \# & NAICS Code & Industry \\
    \hline
    1     & 11    & Agriculture, Forestry, Fishing and Hunting \\
    2     & 211   & Oil and Gas Extraction \\
    3     & 212 and 213 & Mining and Support Activities for Mining \\
    4     & 22    & Utilities \\
    5     & 23    & Construction \\
    6     & 32411 & Petroleum Refineries \\
    7     & 336   & Transportation Equipment Manufacturing \\
    8     & 3391  & Medical Equipment and Supplies Manufacturing \\
    9     & Other codes in 31-33 & Manufacturing \\
    10    & 42    & Wholesale Trade \\
    11    & 44-45 & Retail Trade \\
    12    & 48-49 & Transportation and Warehousing \\
    13    & 51    & Information \\
    14    & 52    & Finance and Insurance \\
    15    & 53    & Real Estate and Rental and Leasing \\
    16    & 54    & Professional, Scientific, and Technical Services \\
    17    & 55    & Management of Companies and Enterprises \\
    18    & 56    & Administrative and Support and Waste Management and Remediation Services \\
    19    & 61    & Educational Services \\
    20    & 62    & Health Care and Social Assistance \\
    21    & 71    & Arts, Entertainment, and Recreation \\
    22    & 72    & Accommodation and Food Services \\
    23    & 81    & Other Services (except Public Administration) \\
    24    & 92    & Public Administration \\
      \hline
    \hline
    \end{tabular}%
  \label{tab:prod_ind}%
\end{table}%

When moving across these data sources, we try to retain the finest level of detail with regard to industry classification.  In cases where we cannot, we apply the most detailed industry information we can across the sub-classifications.  However, to maintain notational consistency, we refer to the industry with the subscript $m$, even if the industry category level differs.

\section{Places to references}

Check out the NIPA primer to be sure of definitions: \href{http://www.bea.gov/national/pdf/nipa\_primer.pdf}{http://www.bea.gov/national/pdf/nipa\_primer.pdf}.  \href{http://www.bea.gov/iTable/}{NIPA Tables}.

This JCT document might be helpful, but doesn't have a lot on calibration: \href{http://www.jct.gov/x-105-03.pdf}{http://www.jct.gov/x-105-03.pdf}.


\bibliography{/Users/jasondebacker/repos/dynamic/Data/Calibration/TaxModelCalibrationReferences}

\end{document}
