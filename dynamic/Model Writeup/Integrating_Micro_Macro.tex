\documentclass[letterpaper,11pt]{article}

\usepackage{threeparttable}
\usepackage{geometry}
\geometry{letterpaper,tmargin=1in,bmargin=1in,lmargin=1.25in,rmargin=1.25in}
\usepackage[format=hang,font=normalsize,labelfont=bf]{caption}
\usepackage{amsmath}
\usepackage{multirow}
\usepackage{array}
\usepackage{delarray}
\usepackage{amssymb}
\usepackage{amsthm}
\usepackage{enumerate}
\usepackage{lscape}
\usepackage{natbib}
\usepackage{setspace}
\usepackage{float,color}
\usepackage[pdftex]{graphicx}
\usepackage{pdfsync}
\usepackage{verbatim}
\usepackage{placeins}
\synctex=1
\usepackage{hyperref}
\hypersetup{colorlinks,linkcolor=red,urlcolor=blue,citecolor=red}
\usepackage{bm}
\usepackage{soul}


\theoremstyle{definition}
\newtheorem{theorem}{Theorem}
\newtheorem{acknowledgement}[theorem]{Acknowledgement}
\newtheorem{algorithm}[theorem]{Algorithm}
\newtheorem{axiom}[theorem]{Axiom}
\newtheorem{case}[theorem]{Case}
\newtheorem{claim}[theorem]{Claim}
\newtheorem{conclusion}[theorem]{Conclusion}
\newtheorem{condition}[theorem]{Condition}
\newtheorem{conjecture}[theorem]{Conjecture}
\newtheorem{corollary}[theorem]{Corollary}
\newtheorem{criterion}[theorem]{Criterion}
\newtheorem{definition}{Definition} % Number definitions on their own
\newtheorem{derivation}{Derivation} % Number derivations on their own
\newtheorem{example}[theorem]{Example}
\newtheorem{exercise}[theorem]{Exercise}
\newtheorem{lemma}[theorem]{Lemma}
\newtheorem{notation}[theorem]{Notation}
\newtheorem{problem}[theorem]{Problem}
\newtheorem{proposition}{Proposition} % Number propositions on their own
\newtheorem{remark}[theorem]{Remark}
\newtheorem{solution}[theorem]{Solution}
\newtheorem{summary}[theorem]{Summary}
\bibliographystyle{aer}
\newcommand\ve{\varepsilon}
\renewcommand\theenumi{\roman{enumi}}
\newcommand\norm[1]{\left\lVert#1\right\rVert}

\begin{document}

\title{Integrating the Microsimulation and Macro Tax Models}
\maketitle

\begin{abstract}
\small{This note outlines the interaction between OSPC's dynamic scoring model and its microsimulation model.  In particular, we summarize the variables and parameters that are passed between the two and discuss how the macro forecast is determined under alternative policies.}
\end{abstract}

\ \\
\ \\

OSPC is developing both a microsimulation model and a general equilibrium macroeconomic model to forecast tax policy changes.  The strength of the former is that it accounts for details of the tax code and forms revenue estimates based on the economic behavior of a large number of individuals represented in the IRS public use files.  The strength of the later is to account for supply side tax policies and general equilibrium effects.  A large contribution of OSPC's tax scoring projects will be the integration of the two models.  To the best of our knowledge, no organization has successfully wed a micro-founded dynamic general equilibrium model to a microsimulation model to evaluate tax policy.

\section{Arguments to pass from the micro to the macro model}

The dynamic scoring model posits multiple functions that are smooth and dependent upon various state variables in the macro model; e.g., a payroll tax function and an individual income tax function. The state variables that are arguments in these functions include age, labor income, and capital income. The calendar year (or model period) is also a state variable and the tax functions will differ across calendar year as current law or the policy proposal dictate.  Note that there are some limitations on the shape of this function to ensure an equilibrium can be determined.  In particular, total taxes must not be decreasing in income.

To fit these tax functions, the macro model needs data on payroll and income taxes paid by filer age, labor income, capital income, and calendar year. Note that the macro model does not currently account for the structure of the filing unit, so in the case of a filing unit with primary and secondary filers, we determine age by that of the primary filer. Thus, the relevant output from the micro model would be the payroll and income tax burdens (in dollars), age, earned income (which equates to labor income in our model), capital income for each individual in each year used in the microsimulation model. With these data, we can then fit the functions for total payroll and income taxes paid that are used in the macro model.

\section{Arguments to pass from the macro model to the micro model}

The microsimulation model relies on a set of extrapolators to forecast the future revenue consequences of tax policy.  These are generally based on growth rates of macroeconomic aggregates.  Our macroeconomic model will produce annual forecasts for GDP, labor income, and real interest rates.  It also produces forecasts of corporate profits, although firms in the model earn zero economic profits.

The model does lack some key extrapolators for the micro model, namely nominal variables such as the price index and nominal interest rates.  The model does produce a forecast of hours worked the may be used for the micro model, but there is no unemployment in the macro model so a forecast of unemployment rates will not be available.
% Not sure how we include these (or if we do) as of yet.


\begin{table}[htbp]
  \centering
  \caption{Summary of Economic Aggregates from Macro Model}
    \begin{tabular}{lcl}
    \hline
    \hline
    Economic Aggregate & Available from Macro Model & Note \\
    \hline
    Wages & Yes   & Really earned income \\
    Hours worked & Yes   &  \\
    GDP   & Yes   &  \\
    Real interest rates & Yes   &  \\
    Corporate Profits & Yes   & Firms earn zero econ profits \\
    Price index & No    &  \\
    Nominal interest rates & No    &  \\
    Unemployment & No    &  \\
   \hline
   \hline
    \end{tabular}%
  \label{tab:macro_vars}%
\end{table}%


\section{Determining the macro forecast}

The goal for the interaction between the two models is for the revenue estimates from each to be consistent with each other.  In particular, that the data used to calibrate the tax functions of the macro model be the result of the microsimulation model taking as inputs the macro forecast that uses those same data in order to calibrate the tax function.  This represents a fixed point in the process.\footnote{I don't think we can prove that one exists, but it seems plausible.}

To find this fixed point, we use an iterative process.  The algorithm is as follows:
\begin{enumerate}
\item Make an initial guess at the macro tax functions (e.g. based on a prior estimate with similar tax policy parameters)
\item Solve the macro model and find the forecast of economic aggregates
\item Feed these aggregates into the microsimulation model and run the model
\item Using the data on taxes paid output from the micro model, estimate tax functions for the macro model
\item Solve the macro model and find the forecast of economic aggregates
\item Compare the aggregates in (v) to those in (ii).  If they are sufficiently close, stop - a fixed point has been found.  If not, repeat starting at step (iii), using the aggregates determined from (v).
\end{enumerate}

When a fixed point is found, we have macro aggregates that are consistent with the tax policies as simulated in the micro model.  This is the goal of the interaction between the two.

\end{document}