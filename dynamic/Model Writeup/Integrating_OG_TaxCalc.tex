\documentclass[letterpaper,12pt]{article}

\usepackage{threeparttable}
\usepackage{geometry}
\geometry{letterpaper,tmargin=1in,bmargin=1in,lmargin=1.25in,rmargin=1.25in}
\usepackage[format=hang,font=normalsize,labelfont=bf]{caption}
\usepackage{amsmath}
\usepackage{multirow}
\usepackage{array}
\usepackage{delarray}
\usepackage{amssymb}
\usepackage{amsthm}
\usepackage{lscape}
\usepackage{natbib}
\usepackage{setspace}
\usepackage{float,color}
\usepackage[pdftex]{graphicx}
\usepackage{pdfsync}
\usepackage{verbatim}
\usepackage{placeins}
\usepackage{geometry}
\usepackage{pdflscape}
\synctex=1
\usepackage{hyperref}
\hypersetup{colorlinks,linkcolor=red,urlcolor=blue,citecolor=red}
\usepackage{bm}


\theoremstyle{definition}
\newtheorem{theorem}{Theorem}
\newtheorem{acknowledgement}[theorem]{Acknowledgement}
\newtheorem{algorithm}[theorem]{Algorithm}
\newtheorem{axiom}[theorem]{Axiom}
\newtheorem{case}[theorem]{Case}
\newtheorem{claim}[theorem]{Claim}
\newtheorem{conclusion}[theorem]{Conclusion}
\newtheorem{condition}[theorem]{Condition}
\newtheorem{conjecture}[theorem]{Conjecture}
\newtheorem{corollary}[theorem]{Corollary}
\newtheorem{criterion}[theorem]{Criterion}
\newtheorem{definition}{Definition} % Number definitions on their own
\newtheorem{derivation}{Derivation} % Number derivations on their own
\newtheorem{example}[theorem]{Example}
\newtheorem{exercise}[theorem]{Exercise}
\newtheorem{lemma}[theorem]{Lemma}
\newtheorem{notation}[theorem]{Notation}
\newtheorem{problem}[theorem]{Problem}
\newtheorem{proposition}{Proposition} % Number propositions on their own
\newtheorem{remark}[theorem]{Remark}
\newtheorem{solution}[theorem]{Solution}
\newtheorem{summary}[theorem]{Summary}
\bibliographystyle{aer}
\newcommand\ve{\epsilon}
%\renewcommand\theenumi{\roman{enumi}}
\newcommand\norm[1]{\left\lVert#1\right\rVert}

\begin{document}

\title{Integration of the Overlapping Generations Macro Model and the Tax Calculator}
\date{\today}
\author{}
\maketitle

\begin{spacing}{1.5}
\pagenumbering{arabic}


\section*{Introduction}
The goal is to have these two models ``talking" to each other in the sense that the macro forecasts used to extrapolate the micro model are produced from the macro model and the tax functions used in the macro model are derived from the tax rates found in the Tax Calculator.

\section{Using the Tax Calculator to get tax functions for the macro model}
The Tax Calculator uses rich micro data and so allows one to model the effect of many changes in the tax code, from changes to the marginal rate structure to the elimination or expansion of certain credits or deductions.  In contrast, the macro model does not allow for such detail, having households who only differ by age and productivity.  Therefore, to have the OG model account for changes to the tax code that rely on specific household characteristics that are not modeled (such as number of dependents, homeownership, and so forth), we will use the output from the Tax Calculator to inform the OG model.  In particular, we'll take the marginal and average tax rates that the Tax Calculator finds for each individual in the micro data and fit a function to these rates.  The function will be a function of variables included in our macro model such as age and total income.


\subsection{Income axes in the household's problem}
The macro model specifies the following taxes, which enter the household's problem through their budget constraint:

\begin{itemize}
\item Earned income tax:  $T^{l}_{j,s,t}  = \tau^{l}_{s,t}(\hat{y}_{j,s,t})w_t e_{j,s}n_{j,s,t}$, where $\hat{y}$ is stationarized income.
\item Capital income tax: $T^k_{j,s,t}  = \tau^{k}_{s,t}(\hat{y}_{j,s,t})r_{t}a_{j,s,t}$.
\item Net payroll taxes: $T^p_{j,s,t}  = \tau^{p}_{t}(w_t e_{j,s}n_{j,s,t})w_t e_{j,s}n_{j,s,t} - mathbbm{1}_{s\geq R}\{\theta_j w_t\}$
\item Estate taxes: $T^{bq}_{j,t}  =\tau^{bq}_{s,t}(\hat{bq}_{j,s,t})bq_{j,s,t}$, where $bq_{j,s,t}$ total bequests by a household of type $j$ and age $s$ in year $t$.  $\hat{bq}$ is stationarized bequests.
\item Wealth tax: $T^w_{j,s,t} = \tau^{w}_{s,t}(\hat a_{j,s,t}) a_{j,s,t}$, where $a_{j,s,t}$ are total assets owned by a household of type $j$ and age $s$ in year $t$.  $\hat{a}$ is stationarized asset holdings. 
\end{itemize}
Note that all tax functions have a $t$ subscript since they may vary by model year as the tax code changes.  Functions also vary over age, this could be due to difference in the number of dependents or type of capital income across age, or differences in whom assets are bequeathed to by those dying at different ages, for example, due to the likelihood of a living spouse.

\subsubsection{Earned income taxes}
The Tax Calculator can output marginal taxes rates by different types of income sources (and perhaps average tax rates as well, though I'm not sure of that).  We will take the marginal tax rate (or average tax rate) for each individual in the Public Use File (PUF).  We have (imputed) age for these filers in the PUF, as well as their total income.  With this, we estimate a smooth function for marginal tax rates on labor income over age and total income.  We can integrate over this function to get total and average taxes paid by age and income group as well.  These give us the average tax function above, $ \tau^{l}_{s,t}(\hat{y}_{j,s,t})$.

Note that we might be able to estimate a different tax function over income for each age.  Two potential issues here: 1) Amount of data in PUF to do this well and 2) It is a problem in the HH FOCs if taxes aren't smooth over age?

\subsubsection{Capital income taxes}
The methodology for capital income taxes will be similar to that for earned income taxes.  The Tax Calculator is especially important here to help us capture difference in capital income taxes due to household portfolio heterogeneity and capital gains realization behavior.

\subsubsection{Net payroll taxes}

We may not need the Tax Calculator to estimate this at all.  Note that here will have just the employees share.  The employers share will be accounted for in the firm's problem.  Another note here is the earnings cap.  We should account for this kink, but does it cause problems in computing the model solution?

\subsubsection{Estate taxes}

Not sure if the Tax Calculator has anything to say about these.  Maybe we can use just the statutory rates, though might want to make some adjustments to those for behavior.

\subsubsection{Wealth taxes}

Not in current law, but we may want to be able to evaluate.  I'm not sure if the microsim guys have imputed wealth onto the PUF.  If they have, it may be helpful here if we ever consider a wealth tax.

\subsection{Individual income taxes in the firm's problem}

We will need the Tax Calculator to help use find a few things that are helpful to solving the firms problem:
\begin{enumerate}
\item Average effective and marginal tax rates for the ``average" owner of non-corporate businesses.  By average, I think we'll want to take a weighted (over income) average of the tax rates on business income for filers with non-corporate business income.
\item AETR and MTR on corporate dividend income (at individual level) for the ``marginal investor". One way to interpret ``marginal investor" is to find the median dividend income (from those with dividends) and find the rate faced by this filer. 
\item AETR and MTR on capital gains income (at individual level) for the ``marginal investor".  One way to interpret is to find the median capital gains income (from those with capital gains) and find the rate faced by this filer 
\item AETR, MTR on interest income (at individual level) for the ``marginal investor". One way to interpret is to find the median interest income (from those with interest income) and find the rate faced by this filer. 
\end{enumerate}


\subsection{Some important questions}

\begin{enumerate}
\item Marginal vs. average tax rates - which should we use?  On the one hand, you want to nail the marginal tax rates to get the behavioral responses correct. On the other hand, you want to get the average tax rates right to account for the magnitude of the tax change, especially when considering revenue estimates.  These two may be at odds.  One proposal is to use marginal tax rates in the necessary conditions of the model, then use some scaling parameter to hit tax revenue targets.
\item How do we related the model's stationarized income to tax law?  Should we worry about ``real bracket creep" if we are using current law to project forward?  
\item There is no inflation in the macro model, how will we handle certain current law policies (or policy proposals) that are not indexed to inflation?
\item We may not be able to directly access the Public Use File - so we might need our code estimating tax functions to run on another server where this is housed then bring the results to the macro model.
\item  The Tax Calculator can build in some behavioral elasticities.  Do we use that?  If so, we need to not double count the labor supply elasticity which is explicitly accounted for in our OG model.  At the same time, it would be nice to have some reporting and evasion behavior built in since we can't account for that in the OG model.
\item I forget why we have an issue separating earned and capital income taxes before.  We might want to think about what that was and if it still pertains.
\end{enumerate}


\section{Incorporating the macro forecast into the micro model}



\end{spacing}
\end{document}